% !TeX root = ../arara-manual.tex
\chapter*{Prologue}
\label{chap:prologue}

\epigraph{Moral of the story: never read the
documentation, bad things happen.}{\textsc{David Carlisle}}

{\setlength{\parskip}{1em}
Writing software is easy. Writing \emph{good} software is extremely difficult. When the counter stopped at version 3.0, Brent, Marco and I decided it was time for \arara\ to graduate and finally be released in \TeX\ Live. My life had changed.

It was a success. A lot of people liked the idea of explicitly telling our tool how to compile their \TeX\ dcouments instead of relying on guesswork. It was indeed a cool concept! But then, the inevitable happened: a lot of bugs had emerged from the dark depths of my humble code.

In all seriousness, \emph{I was about to give up}. My code was not awful, but there were a couple of critical and blocking bugs. Something very drastic had to be done in order to put \arara\ back on track. Then, walking on faith, I decided to rewrite the tool entirely from scratch. In order to achieve this goal, I created a \href{https://github.com/cereda/nightingale}{sandbox} and started working on the new code.

It was my redemption. Nicola helped me with the new version, writing code, fixing bugs and suggesting new features. Soon, we all achieved a very pleasant result. It was like \arara\ was about to hatch again. Version 4.0 was definitely at our hands. Now, it is up to you.

Surprisingly, this humble user manual is not the best resource for learning about our tool. If you really want to see \arara\ in action, I strongly recommend \href{https://www.dickimaw-books.com/latex/admin}{\LaTeX\ for administrative work}, an amazing book freely available for download. The author is, of course, Nicola herself! She explains how \LaTeX\ can be used for administrative work, such as writing correspondence, performing repetitive tasks or typesetting problem sheets on exam papers. And \arara\ is there!

Enjoy the new version. Happy \TeX ing with \arara!
\par}

\vfill

\begin{flushright}
Paulo Roberto Massa Cereda\\
\emph{on behalf of the \arara\ team}
\end{flushright}
