% !TeX root = ../arara-manual.tex
\chapter{Deploying the tool}
\label{chap:deployingthetool}

% TODO fix reference
As previously mentioned, \arara\ runs on top of a Java virtual machine, available on all major operating systems -- in some cases, you might need to install the proper virtual machine. This chapter provides detailed instructions on how to properly deploy the tool in your computer from either the official package available in our project repository or a personal build generated from source (as seen in Section~\ref{foo}, on page~\pageref{foo}).

\begin{messagebox}{No more installers}{araracolour}{\icok}{white}
Be mindful that, from version 4.0 on, the team decided to not release cross-platform installers any more. Our tool is available out of the shelf on all major \TeX\ distributions, including \TeX\ Live and MiK\TeX, which makes manual installation unnecessary given the significant coverage of such distributions. Chances are you already have \arara\ in your system!
\end{messagebox}

\section{Directory structure}
\label{sec:directorystructure}

% TODO fix reference
From the early development stages, our tool employs a very straightforward directory structure. In short, we provide the \abox[araracolour]{ARARA\_HOME} alias to the directory path in which the \rbox[araracolour]{arara.jar} Java archive file is located. This particular file is the heart and soul of our tool and dictates the default rule search path, which is a special directory named \abox[araracolour]{rules/} available from the same level. This directory contains all rules specified in the YAML format, as seen in Section~\ref{foo}, on page~\pageref{foo}. The structure overview is presented as follows.

\vspace{1em} 

{\centering\begin{forest}
for tree={
  grow'=0,
  edge={araracolour},
  anchor=base west
},
forked edges,
[{\abox[araracolour]{ARARA\_HOME}}
  [{\rbox[araracolour]{arara.jar}}]
  [{\abox[araracolour]{rules/}},s sep=1mm
    [{\rbox[araracolour]{animate.yaml}}]
    [{\rbox[araracolour]{bib2gls.yaml}}]
    [{\color{araracolour}\ldots},no edge]
    [{\rbox[araracolour]{xetex.yaml}}]
    [{\rbox[araracolour]{xindy.yaml}}]
  ]
]
\end{forest}\par}

\vspace{1.4em}

Provided that this specific directory structure is honoured, the tool is ready for use out of the shelf. In fact, the official \arara\ package available in the \href{https://github.com/cereda/arara/releases}{release section} of our project repository, as well as the \href{https://bintray.com/cereda/arara}{Bintray} software distribution service, exactly mirrors this structure. Once the package is properly downloaded, we simply need to extract it into a proper \abox[araracolour]{ARARA\_HOME} location.

\section{Defining a location}
\label{sec:definingalocation}

First and foremost, we need to obtain \rbox{arara-4.0.zip} from either our project repository at GitHub or at the Bintray service mirror. As the name indicates, this is a compressed file format, so we need to extract it into a proper location. Run the following command in the terminal:

\begin{codebox}{Terminal}{teal}{\icnote}{white}
$ unzip arara-4.0.zip
\end{codebox}

As a result of the previous command, we obtained a directory named \abox[araracolour]{arara} with the exact structure presented in Section~\ref{foo} in our working directory. Now we need to decide where \arara\ should reside in your system. For example, I usually deploy my tools inside the \abox[araracolour]{/opt/paulo} path, so I need to run the following command in the terminal (please note that my personal directory already has the proper permissions, so I do not need superuser privileges):

\begin{codebox}{Terminal}{teal}{\icnote}{white}
$ mv arara /opt/paulo/
\end{codebox}

The tool has found a comfortable home inside my system! Observe that the full path of the \abox[araracolour]{ARARA\_HOME} reference points out to \abox[araracolour]{/opt/paulo/arara} since this is my deployment location of choice. The resulting structure overview, from the root directory, is presented as follows:

\vspace{1em} 

{\centering\begin{forest}
for tree={
  grow'=0,
  edge={araracolour},
  anchor=base west
},
forked edges,
[{\abox[araracolour]{/}},s sep=1mm
  [{\abox[araracolour]{bin/}}]
  [{\abox[araracolour]{boot/}}]
  [{\color{araracolour}\ldots},no edge]
  [{\abox[araracolour]{opt/}},s sep=1mm
    [{\abox[araracolour]{paulo/}}
      [{\abox[araracolour]{arara/}}
        [{\rbox[araracolour]{arara.jar}}]
        [{\abox[araracolour]{rules/}},s sep=1mm
          [{\rbox[araracolour]{animate.yaml}}]
          [{\rbox[araracolour]{bib2gls.yaml}}]
          [{\color{araracolour}\ldots},no edge]
          [{\rbox[araracolour]{xetex.yaml}}]
          [{\rbox[araracolour]{xindy.yaml}}]
        ]
      ]
    ]
    [{\color{araracolour}\ldots},no edge]
    [{\abox[araracolour]{texbin/}}]
  ]
  [{\color{araracolour}\ldots},no edge]
  [{\abox[araracolour]{usr/}}]
  [{\abox[araracolour]{var/}}]
]
\end{forest}\par}

\vspace{1.4em}

If you built our tool from source (as indicated in Section~\ref{foo}, on page~\pageref{foo}), make sure to construct the provided directory structure previously presented. We can test the deployment by running the following command in the terminal (please note the full path):

\begin{codebox}{Terminal}{teal}{\icnote}{white}
$ java -jar /opt/paulo/arara/arara.jar
  __ _ _ __ __ _ _ __ __ _ 
 / _` | '__/ _` | '__/ _` |
| (_| | | | (_| | | | (_| |
 \__,_|_|  \__,_|_|  \__,_|

arara 4.0 (revision 1)
Copyright (c) 2012-2018, Paulo Roberto Massa Cereda
All rights reserved

usage: arara [file [--dry-run] [--log] [--verbose | --silent]
             [--timeout N] [--max-loops N] [--language L]
             [ --preamble P ] [--header] | --help | --version]
 -h,--help                 print the help message
 -H,--header               extract directives only in the file header
 -l,--log                  generate a log output
 -L,--language <code>      set the application language
 -m,--max-loops <number>   set the maximum number of loops
 -n,--dry-run              go through all the motions of running a
                           command, but with no actual calls
 -p,--preamble <name>      set the file preamble based on the
                           configuration file
 -s,--silent               hide the command output
 -t,--timeout <number>     set the execution timeout (in milliseconds)
 -V,--version              print the application version
 -v,--verbose              print the command output
\end{codebox}

Please observe that, provided that the underlying operating system has an appropriate Java virtual machine installed, \arara\ can be used as a portable, standalone application. Portable applications can be stored on any data storage device, including internal mass storage, a file share, cloud storage or external storage such as USB drives and floppy disks.

\section{Linking the tool}
\label{sec:linkingthetool}



%\begin{}{Source file}{teal}{\icnote}{white}{}
%\end{ncodebox}

%\begin{codebox}{}{teal}{\icnote}{white}
%\end{codebox}

%\begin{messagebox}{}{araracolour}{\icok}{white}
%\end{messagebox}
